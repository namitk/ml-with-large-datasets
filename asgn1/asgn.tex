\documentclass[paper=a4, fontsize=11pt]{scrartcl} % A4 paper and 11pt font size

\usepackage[T1]{fontenc} % Use 8-bit encoding that has 256 glyphs
\usepackage{utopia} % Use the Adobe Utopia font for the document - comment this line to return to the LaTeX default
\usepackage[english]{babel} % English language/hyphenation
\usepackage{amsmath,amsfonts,amsthm} % Math packages
\usepackage{sectsty} % Allows customizing section commands
\allsectionsfont{\centering \normalfont\scshape} % Make all sections centered, the default font and small caps
\usepackage{fancyhdr} % Custom headers and footers
\pagestyle{fancyplain} % Makes all pages in the document conform to the custom headers and footers
\fancyhead{} % No page header - if you want one, create it in the same way as the footers below
\fancyfoot[L]{} % Empty left footer
\fancyfoot[C]{} % Empty center footer
\fancyfoot[R]{\thepage} % Page numbering for right footer
\renewcommand{\headrulewidth}{0pt} % Remove header underlines
\renewcommand{\footrulewidth}{0pt} % Remove footer underlines
\setlength{\headheight}{13.6pt} % Customize the height of the header
\numberwithin{equation}{section} % Number equations within sections (i.e. 1.1, 1.2, 2.1, 2.2 instead of 1, 2, 3, 4)
\numberwithin{figure}{section} % Number figures within sections (i.e. 1.1, 1.2, 2.1, 2.2 instead of 1, 2, 3, 4)
\numberwithin{table}{section} % Number tables within sections (i.e. 1.1, 1.2, 2.1, 2.2 instead of 1, 2, 3, 4)

\setlength\parindent{0pt} % Removes all indentation from paragraphs - comment this line for an assignment with lots of text
% \addtolength{\oddsidemargin}{-.15in}
% \addtolength{\evensidemargin}{-0.15in}
% \addtolength{\textwidth}{-0.5in}
\addtolength{\topmargin}{-1.0in}
\addtolength{\textheight}{1.5in}


\newcommand{\horrule}[1]{\rule{\linewidth}{#1}} % Create horizontal rule command with 1 argument of height

\title{	
\normalfont \normalsize 
% \textsc{university, school or department name} \\ [25pt] % Your university, school and/or department name(s)
\horrule{0.5pt} \\[0.4cm] % Thin top horizontal rule
\huge{10-605 Assignment 1} \\ % The assignment title
\horrule{2pt} \\[0.5cm] % Thick bottom horizontal rule
}

\author{Namit Katariya (andrew id: nkatariy)} % Your name
\date{\normalsize\today} % Today's date or a custom date

\begin{document}

\maketitle 

\section*{\textbf{Question 1}}

\begin{itemize}
\item \underline{Output of training and testing on the "Very Small" dataset}:
\begin{verbatim}
[C24,CCAT,M14,MCAT]                      GCAT     -8837.889580619143
[E51,E512,ECAT,GCAT,GDIP]                CCAT     -3321.9899605306355
[C15,C152,C18,C181,CCAT]	                CCAT	    -842.0977201801019
[GCAT]                                   CCAT     -1393.8777561662928
[C13,CCAT,GCAT,GHEA]                     CCAT     -604.5813059774287
[C13,CCAT,M11,MCAT]                      CCAT     -1261.9526299152496
[C11,C13,CCAT,E12,ECAT,M13,M132,MCAT]    CCAT     -1867.783474781556
[C31,CCAT]                               CCAT     -1909.4938087854025
\end{verbatim}
\item
\begin{tabular}{|c | c|}
\hline
\textbf{Dataset} & \textbf{Percent Correct} \\
\hline
Very Small & 62.5 \\
Small & 80.44 \\
Full & 84.49 \\
\hline
\end{tabular}
\end{itemize}

\section*{\textbf{Question 2a}}
One could discard frequently occurring words like ``the'', ``a'', ``for'' etc. A better way would be to do what we discussed in class : instead of actually incrementing variables, we output statements corresponding to these increments and write it to a file. We can retrieve the counts by sorting this file and collecting the terms occurring. 

\section*{\textbf{Question 2b}}
One could keep a threshold value between 0 and 1 and predict the document to have all the labels which have a probability above this threshold. The threshold could be decided differently for different documents. For example, if the probabilities turn out to be 0.3, 0.2, 0.19, 0.15 and say more than 10 other labels each with probability around 0.02, then 0.1 would be a good threshold value. \\
Another method could be to have a fixed $k$, sort the labels according to their probabilities and label the document as having the top $k$ labels. 

\end{document}